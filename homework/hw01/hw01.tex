\documentclass[12pt]{article}
\usepackage[utf8]{inputenc}
\usepackage[margin=1in, top=1.5in]{geometry}
\usepackage{fancyhdr, amsthm, amssymb, amsmath}
\usepackage{tikz,mathtools}
\usepackage{mdframed}
\usepackage{fancyhdr}
\usepackage{ifthen}
\usepackage{comment}
\usepackage{mathtools}
\usepackage{enumitem}
\usepackage{tikz}
\usepackage{tikz-cd}

\newboolean{solutions}
\setboolean{solutions}{true}
\ifthenelse{\boolean{solutions}}
{\newenvironment{solution}{\begin{mdframed}[skipbelow=0pt, linecolor=White, backgroundcolor=Green!10]\textbf{Solution:}}{\end{mdframed}}}
{\excludecomment{solution}}

\setlength{\parindent}{0pt}

\newenvironment{problem}[2][Problem]{\begin{trivlist}
     \item[\hskip \labelsep {\bfseries #1}\hskip \labelsep {\bfseries #2.}]\bfseries}
{\end{trivlist}}

\setlength\parindent{0pt}
\pagestyle{fancy}
\fancyhf{}
\lhead{Analysis I}
\rhead{Alex Bellon}
\rfoot{\thepage}

% Commonly used commands get ``macros'', which are shorter and more convenient.
\newcommand{\D}{\mathbb{D}}
\newcommand{\N}{\mathbb{N}}
\newcommand{\Z}{\mathbb{Z}}
\newcommand{\Q}{\mathbb{Q}}
\newcommand{\C}{\mathbb{C}}
\newcommand{\R}{\mathbb{R}}
\newcommand{\F}{\mathbb{F}}
\newcommand{\Int}{\operatorname{Int}}

% Specify the environments (definitions, theorems, exercises, examples, etc.) used:
\newtheorem{thm}{Theorem}[section]
\newtheorem{exercise}[thm]{Exercise} % The numbering and style of exercises and theorems should be the same.
\newtheorem{ax}{Axiom}
% The asterisk means it will not be numbered.
\newtheorem*{defn}{Definition}
\newtheorem*{example}{Example}
\newtheorem*{examples}{Examples}

\begin{document}

\section*{Homework 1}

\begin{problem}{2.2.2}
Let $a$ be a positive number. Prove there exists exactly one natural number $b$ such that $b++ = a$.
\end{problem}

\begin{proof}
  Proof by contradiction. Assume natural numbers $b \neq c$ such that $b++ = a$ and $c++ = a$. From Axiom 2.4, we know that if $b \neq c$, then $n++ \neq m++$. This implies that $a \neq a$, which is not true.
  % Do I have transitivity? Can I even use proof by contradiction? Do I have reflexivity?
\end{proof}

\begin{problem}{3.1.6}
Let $A, B, C$ be sets, and let $X$ be a set containing $A, B, C$ as subsets. Prove De Morgan's law for sets.
\begin{enumerate}
    \item $X \setminus (A \cup B) = (X \setminus A) \cap (X \setminus B)$
    \item $X \setminus (A \cap B) = (X \setminus A) \cup (X \setminus B)$
\end{enumerate}
Parts a-g of proposition 3.1.27 can be assumed.
\end{problem}

\begin{proof}
.
\begin{enumerate}
  \item Let $X = \{A \cup B \cup C \cup Y\}, Y$, where Y is everything else in $X$ that is not $A \cup B \cup C$. Then
  \begin{align*}
    X \setminus (A \cup B) &= C\setminus (A\cup B) \cup Y\\
    X \setminus A &= B\setminus A \cup C\setminus A \cup Y\\
    X \setminus A &= B\setminus A \cup C\setminus A \cup Y\\
  \end{align*}

\end{enumerate}
\end{proof}

\begin{problem}{3.3.2}
Let $f: X \to Y$ and $g: Y \to Z$ be functions. Show that if $f$ and $g$ are both injective, then so is $g \circ f$; similarly show that if $f$ and $g$ are both surjective, then so is $g \circ f$.
\end{problem}

\begin{proof}

\end{proof}

\begin{problem}{3.4.2}
Let $f: X \to Y$ be a function from one set $X$ to another set $Y$, let $S$ be a subset of $X$, and let $U$ be a subset of $Y$. What, in general can one say about $f^{-1}(f(S))$ and $S$. What about $f(f^{-1}(U))$ and $U$. Prove your assertions.
\end{problem}

\begin{proof}

\end{proof}

\begin{problem}{3.5.10}
If $f: X \to Y$ is a function, define the \emph{graph} of $f$ to be the subset of $X \times Y$ defined by $\{x, f(x)\} : x \in X$. Show that two functions $f: X \to Y$. $\bar{f}: X \to Y$ are equal if and only if they share the same graph. Conversely, if $G$ is any subset of $X \times Y$ with the property that for each $x \in X$, the set $\{y \in Y : (x,y) \in G\}$ has exactly one element (or in other words, $G$ obeys the vertical line test), show that there is exactly one function $f: X \to Y$ whose graph is equal to $G$.
\end{problem}

\begin{proof}

\end{proof}

\begin{problem}{3.6.10}
Let $A_1, \cdots, A_n$ be finite sets such that $\#(\bigcup_{i \in \{1, \cdots, n\}} A_i) > n$. Show that there exists $i \in \{i, \cdots, n\}$ such that $\#(A_i) \ge 2$. (This is known as the pidgeonhole principle.)
\end{problem}

\begin{proof}

\end{proof}

\end{document}
