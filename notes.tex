\documentclass[11pt]{article}
\usepackage[utf8]{inputenc}
\usepackage[margin=.8in, top=1.2in]{geometry}
\usepackage{fancyhdr, amsthm, amssymb, amsmath}
\usepackage{tikz,mathtools}

\setlength\parindent{0pt}
\pagestyle{fancy}
\fancyhf{}
\lhead{Analysis I}
\rhead{Notes}
\rfoot{\thepage}

\begin{document}

\section{Introduction}

\subsection{What is analysis?}

\begin{itemize}
  \item \textbf{Analysis}: the rigorous study of objects such as real numbers, complex numbers, etc.
\end{itemize}

\subsection{Why do analysis?}

\begin{itemize}
  \item To understand how these objects work so that you can apply them intelligently
\end{itemize}

\section{The Natural Numbers}

\subsection{The Peano Axioms}

\begin{itemize}
  \item \textbf{Axiom 2.1} 0 is a natural number
  \item \textbf{Axiom 2.2} If $n$ is a natural number, then $n++$ is also a natural number
  \item \textbf{Axiom 2.3} 0 is not the successor of any natural number
  \begin{itemize}
    \item This prevents ``wrap-around"
  \end{itemize}
  \item \textbf{Axiom 2.4} Different natural numbers must have different successors: if $n, m$ are natural numbers and $n \neq m$, then $n++ \neq m++$. Equivalently, if $n++ = m++$ then $n = m$
    \begin{itemize}
      \item This prevents incrementation hitting a ``ceiling"
    \end{itemize}
  \item \textbf{Axiom 2.5 (Principle of mathematical induction)} Let $P(n)$ be any property pertaining to a natural number $n$. Suppose that $P(0)$ is true, and suppose that whenever $P(n)$ is true, $P(n++)$ is also true. Then $P(n)$ is true $\forall$ natural $n$.
  \begin{itemize}
    \item This prevents ``rogue" elements (like rational numbers) from being in the naturals
  \end{itemize}
  \item This definition is \textit{axiomatic} and not \textit{constructive}, they lay out what you can do with the naturals and properties they have, rather than what they are
\end{itemize}

\subsection{Addition}

\begin{itemize}
  \item \textbf{Def 2.2.1 (Addition)} Let $m$ be a natural number. To add zero to $m$, we define $0 + m \coloneqq m$. Now by induction, suppose we know how to add $n$ to $m$. Adding $n++$ to $m$ can the be defined as
  \[
    (n++)+m \coloneqq (n+m)++
  \]
  \item \textbf{Lemma 2.2.2} For any natural number $n$, $n+0 = n$
  \begin{itemize}
    \item This cannot be proven from $0 + m = m$, as we haven't proven commutativity
  \end{itemize}
  \item \textbf{Lemma 2.2.3} For any natural numbers $n$ and $m$, $n+(m++) = (n+m)++$.
  \item \textbf{Proposition 2.2.4 (Addition is commutative)} For any natureal numbers $n,m$, $n+m = m + n$
  \item \textbf{Proposition 2.2.5 (Addition is associative)} For any natural numbers $a,b,c$, we have $(a+b)+c = a+(b+c)$
  \item \textbf{Proposition 2.2.6 (Cancellation law)} Let $a,b,c$ be natural numbers such that $a+b = a+c$. Then we have that $b=c$
  \item \textbf{Definition 2.2.7 (Positive natural numbers)} A natural number $n$ is positive iff it is not equal to 0
  \item \textbf{Proposition 2.2.8} If $a$ is positive and $b$ is a natural number, then $a+b$ is positive
  \item \textbf{Corollary 2.2.9} If $a$ and $b$ are natural numbers such that $a+b = 0$, then $a=0$ and $b=0$
  \item \textbf{Lemma 2.2.10} Let $a$ be a positive number. Then there exists only one natural number $b$ such that $b++ = a$
  \item \textbf{Definition 2.2.11 (Ordering of the natural numbers)} Let $n,m$ be natural numbers. We say that $n$ is greater than or equal to $m$ if $n = m + a$ for some natural number $a$
  \item \textbf{Proposition 2.2.12} Order is
  \begin{enumerate}
    \item reflexive
    \item transitive
    \item anti-symmetric ($a \geqslant b, b \geqslant a \implies a =b$)
    \item preserved by addition
    \item $a < b$ iff $a++ \leqslant b$
    \item $a < b$ iff $b = a + d$ for some positive number $d$
  \end{enumerate}
  \item \textbf{Proposition 2.2.13} Let $a,b$ be natural numbers. Either $a < b, a=b$ or $a > b$
\end{itemize}

\subsection{Multiplication}

\begin{itemize}
  \item \textbf{Definition 2.3.1}
  \item \textbf{Lemma 2.3.2}
  \item \textbf{Lemma 2.3.3}
  \item \textbf{Proposition 2.3.4}
  \item \textbf{Proposition 2.3.5}
  \item \textbf{Proposition 2.3.6}
  \item \textbf{Corollary 2.3.7}
  \item \textbf{Proposition 2.3.9}
  \item \textbf{Definition 2.3.11}
\end{itemize}

\section{Set Theory}

\subsection{Fundamentals}
\begin{itemize}
  \item Sets are objects
  \begin{itemize}
    \item Pure set theory does not deal with objects, instead defining them in terms of sets
  \end{itemize}
  \item Sets with ``repeat" elements are still equal to the same set without the repeated element
\end{itemize}

\end{document}
