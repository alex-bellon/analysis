\documentclass[12pt]{article}
\usepackage[utf8]{inputenc}
\usepackage[margin=1in, top=1.5in]{geometry}
\usepackage{fancyhdr, amsthm, amssymb, amsmath}
\usepackage{tikz,mathtools}

\setlength\parindent{0pt}
\pagestyle{fancy}
\fancyhf{}
\lhead{Analysis I}
\rhead{Notes}
\rfoot{\thepage}

\begin{document}

\section{Introduction}

\subsection{What is analysis?}

\begin{itemize}
  \item \textbf{Analysis}: the rigorous study of objects such as real numbers, complex numbers, etc.
\end{itemize}

\subsection{Why do analysis?}

\begin{itemize}
  \item To understand how these objects work so that you can apply them intelligently
\end{itemize}

\section{The Natural Numbers}

\subsection{The Peano Axioms}

\begin{itemize}
  \item \textbf{Axiom 2.1} 0 is a natural number
  \item \textbf{Axiom 2.2} If $n$ is a natural number, then $n++$ is also a natural number
  \item \textbf{Axiom 2.3} 0 is not the successor of any natural number
  \begin{itemize}
    \item This prevents ``wrap-around"
  \end{itemize}
  \item \textbf{Axiom 2.4} Different natural numbers must have different successors: if $n, m$ are natural numbers and $n \neq m$, then $n++ \neq m++$. Equivalently, if $n++ = m++$ then $n = m$
    \begin{itemize}
      \item This prevents incrementation hitting a ``ceiling"
    \end{itemize}
  \item \textbf{Axiom 2.5 (Principle of mathematical induction)} Let $P(n)$ be any property pertaining to a natural number $n$. Suppose that $P(0)$ is true, and suppose that whenever $P(n)$ is true, $P(n++)$ is also true. Then $P(n)$ is true $\forall$ natural $n$.
  \begin{itemize}
    \item This prevents ``rogue" elements (like rational numbers) from being in the naturals
  \end{itemize}
  \item This definition is \textit{axiomatic} and not \textit{constructive}, they lay out what you can do with the naturals and properties they have, rather than what they are
\end{itemize}

\subsection{Addition}

\begin{itemize}
  \item \textbf{Def 2.2.1 (Addition)} Let $m$ be a natural number. To add zero to $m$, we define $0 + m \coloneqq m$. Now by induction, suppose we know how to add $n$ to $m$. Adding $n++$ to $m$ can the be defined as
  \[
    (n++)+m \coloneqq (n+m)++
  \]
  \item \textbf{Lemma 2.2.2} For any natural number $n$, $n+0 = n$
  \begin{itemize}
    \item This cannot be proven from $0 + m = m$, as we haven't proven commutativity
  \end{itemize}
\end{itemize}

\end{document}
